\documentclass[10pt,t]{beamer}

\usepackage{amsmath}
\usepackage{amssymb}
\usepackage{color}
\usepackage{tikz}
\usepackage{pgf}
\usepackage{multimedia} 
\usepackage{array}
\usetikzlibrary{arrows,shapes}

\definecolor{mytitle_fg}{rgb}{0.95,0.95,0.00}
\definecolor{myfootLeft_fg}{rgb}{0.95,0.95,0.90}
\definecolor{myfootLeft_bg}{rgb}{0.0,0.0,0.34}
\setbeamercolor{myfootLeft_col}{fg=myfootLeft_fg,bg=myfootLeft_bg}
\definecolor{myfootCenter_fg}{rgb}{0.95,0.95,0.90}
\definecolor{myfootCenter_bg}{rgb}{0.0,0.0,0.34}
\setbeamercolor{myfootCenter_col}{fg=myfootCenter_fg,bg=myfootCenter_bg}
\definecolor{myfootRight_fg}{rgb}{0.95,0.95,0.90}
\definecolor{myfootRight_bg}{rgb}{0.0,0.0,0.34}
\setbeamercolor{myfootRight_col}{fg=myfootRight_fg,bg=myfootRight_bg}
\definecolor{mathrgb}{rgb}{0.0,0.0,0.74}

%%%%%%%%%%%%%%%%%%%%%%%%%%%%%%%%%%%%%%%%%%%%%
\makeatletter
\long\def\beamer@@frametitle[#1]#2{%
  \beamer@ifempty{#2}{}{%
    \gdef\insertframetitle{\centering{#2\ifnum\beamer@autobreakcount>0\relax{}\space\usebeamertemplate*{frametitle continuation}\fi}}%
  \gdef\beamer@frametitle{#2}%
  \gdef\beamer@shortframetitle{#1}%
}%
}
\makeatother

\usecolortheme[rgb={0.0,0.0,0.26}]{structure}
\useoutertheme[width=0pt,height=11mm]{sidebar}

%from \useinnertheme{rectangles}
\setbeamertemplate{sections/subsections in toc}[square]
\setbeamertemplate{items}[square]

%from \usecolortheme{whale}
\setbeamercolor*{palette primary}{use=structure,fg=mytitle_fg,bg=structure.fg}
\setbeamercolor*{palette secondary}{use=structure,fg=white,bg=structure.fg!75!black}
\setbeamercolor*{palette tertiary}{use=structure,fg=white,bg=structure.fg!50!black}
\setbeamercolor*{palette quaternary}{fg=white,bg=black}
\setbeamercolor*{sidebar}{use=structure,bg=structure.fg}
\setbeamercolor*{palette sidebar primary}{use=structure,fg=structure.fg!10}
\setbeamercolor*{palette sidebar secondary}{fg=white}
\setbeamercolor*{palette sidebar tertiary}{use=structure,fg=structure.fg!50}
\setbeamercolor*{palette sidebar quaternary}{fg=white}
\setbeamercolor*{titlelike}{parent=palette primary}
\setbeamercolor*{separation line}{}
\setbeamercolor*{fine separation line}{}


\setbeamersize{text margin left=4mm, text margin right=4mm} 
\setbeamercolor*{frametitle}{parent=palette primary}
\setbeamerfont{block title}{size={}}
%\setbeamertemplate{frametitle}[default][center]



%\setbeamertemplate{items}[triangle] 
%\setbeamertemplate{blocks}[rounded][shadow=true] 
%\setbeamertemplate{navigation symbols}{} 
\setbeamerfont{frametitle}{size=\Large}
\setbeamercolor{frametitle}{fg=mytitle_fg}

\definecolor{boxrgb}{rgb}{0.7,0.9,0.7}
\setbeamercolor{mybox1}{fg=black,bg=boxrgb}
\setbeamercolor{mybox2}{fg=red,bg=boxrgb}
 
\setbeamertemplate{footline}{\hfill{\insertframenumber\hspace*{2em}}}
\setbeamercolor{footline}{fg=white,bg=blue}
\setbeamertemplate{footline}
{%
  \hbox{
\begin{beamercolorbox}[wd=.20\paperwidth,ht=2.5ex,dp=1.125ex,left]{myfootLeft_col}%
    \usebeamerfont{author in head/foot}~~~\insertshortauthor~
  \end{beamercolorbox}%
  \begin{beamercolorbox}[wd=.6\paperwidth,ht=2.5ex,dp=1.125ex,center]{myfootLeft_col}%
    \usebeamerfont{title in head/foot} http://faculty.cua.edu/plaku, plaku@cua.edu
  \end{beamercolorbox}%
  \begin{beamercolorbox}[wd=.2\paperwidth,ht=2.5ex,dp=1.125ex,right]{myfootLeft_col}%
    \usebeamerfont{title in head/foot}\insertframenumber~~~
  \end{beamercolorbox}%
}\vskip0pt}

\newcommand{\Acronym}[1]{\ensuremath{{{\texttt{#1}}}}}
\newcommand{\Symbol}[1]{\ensuremath{\mathcal{#1}}}
\newcommand{\Function}[1]{\ensuremath{{\textsc{#1}}}}
\newcommand{\Constant}[1]{\ensuremath{{\texttt{#1}}}}
\newcommand{\Var}[1]{\ensuremath{{{\mathrm{#1}}}}}
\newcommand{\False}{\Constant{false}}
\newcommand{\True}{\Constant{true}}
\newcommand{\Null}{\Constant{null}}
\newcommand{\R}{\ensuremath{\mathbb{R}}}
\newcommand{\AlgoFont}[1]{\footnotesize{#1}}
\newcommand{\RefFont}[1]{\tiny{#1}}


\author{Alex Wallar \& Erion Plaku} 

\newcommand{\LTLNext}{\bigcirc}
\newcommand{\LTLEventually}{\Diamond}
\newcommand{\LTLAlways}{\Box}
\newcommand{\LTLUntil}{\cup}
\newcommand{\LTLRelease}{\Symbol{R}}

\title{{\large Path Planning for Swarms by Combining Probabilistic Roadmaps and Potential Fields}}
%\subtitle{Motion Planning with Kinematics and Dynamics}
\institute{Dept. of Electrical Engineering and Computer Science\\
  Catholic University of America\\\vspace*{15mm}
  \textcolor{blue}{\textbf{http://faculty.cua.edu/plaku}}}

\date{}


\begin{document}


%\AtBeginSection[]{%
%  \begin{frame}<beamer>
%    \frametitle{Outline}
%    \tableofcontents[currentsection]
%  \end{frame}
%  \addtocounter{framenumber}{-1}% If you don't want them to affect the slide number
%}


%\AtBeginSection[] {
%  \begin{frame}
%    \frametitle{Overview}
%    \tableofcontents[currentsection]
%  \end{frame}
%  \addtocounter{framenumber}{-1}
%}

\tikzstyle{every picture}+=[remember picture]
\tikzstyle{na} = [baseline=-.5ex]

\begin{frame}[plain]
  \titlepage
\end{frame}


%\begin{frame}
%\frametitle{Coupling of the Discrete and the Continuous}
%Discrete Logic
%\begin{itemize}
%\item Tasks often involve abstractions into discrete logical actions
%\item Discrete logic determines transitions from one action to another
%\item Each discrete action requires substantial continuous motion planning
%\end{itemize}

%\vspace*{10mm}
%Motion Planning
%\begin{itemize}
%\item Robot must avoid collisions
%\item Robot motion obeys physical constraints
%\item Underlying motion dynamics are often nonlinear
%\end{itemize}

%\pause

%\vspace*{10mm}

%\hfill{\textcolor{mathrgb}{the discrete and the continuous have
%    generally been treated separately}}

%\end{frame}

\begin{frame}
\frametitle{Motion Planning}


\textcolor{blue}{Proposed approach automatically plans the necessary
  motions that enable the robot to reach a desired destination while
  avoiding collisions with obstacles}

\vspace*{2mm}

\movie{\includegraphics[width=\textwidth]{figScenePolys}}{bumpy.wmv}

\vspace*{-2mm}
\href{run:bumpy.wmv}{\footnotesize{\hfill{[movie]}}}
 
\begin{columns}

\column{0.45\textwidth}
Motivated by applications in
\begin{itemize}
\item Navigation
\item Exploration
\item Search-and-Rescue
\end{itemize}

\column{0.45\textwidth}


\end{columns}

\end{frame}


\begin{frame}
\frametitle{Computational Challenges}

\textcolor{blue}{Motion planning requires searching a vast high-dimensional state space
for \emph{dynamically-feasible motions} that avoid collisions}


\vspace*{4mm}

\textcolor{red}{Taking motion dynamics into account  is essential to ensure that
  the planned motions can be followed in the physical world}
 
\begin{itemize}
\item Motion dynamics can be complex 
\item Often involve non-linear ODEs
\end{itemize}


\end{frame}



\begin{frame}
\frametitle{Computational Challenges}

\movie{\includegraphics[width=0.9\textwidth]{figs/figSnake}}{strailer3.wmv}
\href{run:strailer3.wmv}{\footnotesize{[movie]}}

\begin{itemize}
\item Snake-like robot model consists of several links attached to
  each other
\item[] Continuous state consists of
\textcolor{blue}{
$
s = (x, y, \theta_0, v, \psi, \theta_1, \theta_2, \ldots, \theta_N)
$}

\vspace*{1mm}

\item Motion dynamics modeled as a car pulling trailers

\item[] The differential
    equations of motions are
\textcolor{blue}{
\begin{tabular}{lllll}
$\dot{x} = v \cos(\theta_0)$  & 
$\dot{y} = v \sin(\theta_0)$ &
$\dot{\theta_0} = v\tan(\psi)$ &
$\dot{v} = a$  & 
$\dot{\psi} = \omega$  \\
\multicolumn{5}{c}{$\dot{\theta_i} = \frac{v}{d} \left(\prod_{j=1}^{i-1}\cos(\theta_{j-1}
- \theta_j)\right) (\sin(\theta_{i-1}) - \sin(\theta))$} 
\end{tabular}}

\vspace*{1mm}

\item[] Robot controls: \textcolor{mathrgb}{$a$} (acceleration); \hspace{2mm}
  \textcolor{mathrgb}{$\omega$} (rotational velocity  of steering wheel)

\end{itemize}

\end{frame}

\begin{frame}
\frametitle{Related Work}

\begin{columns}[c]

\column{0.50\textwidth}

\only<1>{\includegraphics[width=2.5in]{figs/figTreeExp0}}
\only<2>{\includegraphics[width=2.5in]{figs/figTreeExp1}}
\only<3>{\includegraphics[width=2.5in]{figs/figTreeExp2}}
\only<4>{\includegraphics[width=2.5in]{figs/figTreeExp3}}
\only<5->{\includegraphics[width=2.5in]{figs/figTreeExp}}


\includegraphics[width=2.5in]{figs/figDynSim}

\column{0.50\textwidth}

\textcolor{blue}{Sampling-Based Motion Planning}
\vspace*{2mm}

Expand a tree $\Symbol{T}$ of collision-free and dynamically-feasible
motions

\vspace*{2mm}

\begin{itemize}
\item select a state $s$ from which to expand the tree

\vspace*{2mm}

\item sample control input $u$

\vspace*{2mm}

\item generate new trajectory by applying $u$ to $s$
\end{itemize}

\hspace*{3mm}
\textcolor{blue}{Successful Motion Planners}
{\footnotesize{
\begin{itemize}
\item RRT [LaValle, Kuffner: IJRR 2001]
\item EST [Hsu et al: IJRR 2002]
\item PDST [Ladd, Kavraki: RSS 2005]
\item SYCLOP [Plaku, Kavraki, Vardi: IEEE TRO 2010]
\end{itemize}
}}

\end{columns}
\end{frame}

\begin{frame}
\frametitle{Related Work}

\textcolor{blue}{On difficult kinodynamic motion-planning problems it
  has been noted that}


\begin{itemize}
\item Exploration frequently gets stuck
\item Progress slows down
\end{itemize}


\vspace*{5mm}
\begin{itemize}
\item Exploration guided by limited information, such as distance   metrics and nearest neighbors
\item Lack of global sense of direction toward goal
\item Difficult to discover new promising directions toward goal
\end{itemize}

\vspace*{3mm}

{\footnotesize{
[Donald et al., ‘93; Kim et al., ‘05; Plaku et al., ‘05; Yershova et
al., ‘05; Ladd, Kavraki  ‘05; Choset et al., ‘05; LaValle ‘06; Hsu et
al., ‘06; Burns, Brock ‘07; Plaku et al. '10;
Sucan, Kavraki '11]}}

\end{frame}

\begin{frame}
\frametitle{{\large {New Approach}}}

\begin{itemize}
\item {Treats motion planning with dynamics as a search problem in a \emph{hybrid} state space composed of continuous and discrete components}

\vspace*{4mm}

\item  {Couples sampling-based motion planning with discrete search}
\end{itemize}


\vspace*{4mm}

\begin{beamercolorbox}[ht=2.5ex,dp=1.125ex,center]{mybox2}
{\textcolor{blue}{discrete layer}}: guide motion planning
\end{beamercolorbox}

\begin{beamercolorbox}[ht=2.5ex,dp=1.125ex,center]{mybox2}
{\textcolor{blue}{continuous layer}}: expand tree of feasible motions
\end{beamercolorbox}

\begin{beamercolorbox}[ht=2.5ex,dp=1.125ex,center]{mybox2}
{\textcolor{blue}{interplay}}: update guide to reflect motion-planning progress
\end{beamercolorbox}

\vspace*{2mm}

\centerline{\includegraphics[width =
    0.5\textwidth]{figs/figInterplay}}

\vspace*{2mm}

\begin{itemize}
\item  Significantly improves computational efficiency over related work
\end{itemize}

\end{frame}

\begin{frame}
\frametitle{Workspace Decomposition}


\begin{columns}[c]


\column{0.50\textwidth}
\includegraphics[width=\textwidth]{figs/fig1}

\column{0.48\textwidth}
  Workspace decomposition provides discrete layer 
  as adjacency graph 

{\centerline{\textcolor{blue}{$G = (R, E)$}}}

\begin{itemize}
\item $R$ denotes the regions of the decomposition
\item $E = \{(r_i, r_j) : r_i, r_j \in R$ are
  physically adjacent$\}$
\end{itemize}
\end{columns}

\vspace*{25mm}
{\hfill{\includegraphics[width=1.5in]{figs/figInterplayDecomp}}}

\end{frame}

\begin{frame}
\frametitle{Backward Discrete Search to Estimate Region Costs}

 
\begin{columns}[c]

\column{0.50\textwidth}
\includegraphics[width=\textwidth]{figs/fig1}

\column{0.48\textwidth}
  Workspace decomposition provides discrete layer 
  as adjacency graph 

{\centerline{\textcolor{blue}{$G = (R, E)$}}}

\begin{itemize}
\item $R$ denotes the regions of the decomposition
\item $E = \{(r_i, r_j) : r_i, r_j \in R$ are
  physically adjacent$\}$
\end{itemize}
\end{columns}

\vspace*{3mm}

\textcolor{blue}{cost(r)} estimates the difficulty of reaching the goal region from $r$

\hfill{defined as length of shortest path in $G=(R, E)$ from $r$ to goal} 

\vspace*{3mm}
\textcolor{blue}{$[cost(r_1), cost(r_2), \ldots, cost(r_n)]$} 

\hfill{computed by running once A* shortest-path algorithm backwards from goal}


\vspace*{2.7mm}
{\hfill{\includegraphics[width=1.5in]{figs/figInterplayCosts}}}

\end{frame}


\begin{frame}
\frametitle{Region Selection based on Cost Estimates}

\hspace*{-2mm}
At each iteration, a region $r_\Var{from}$ is selected from which to expand the search
tree $T$ \hspace*{-3mm}
 
\begin{columns}[c]


\column{0.45\textwidth}
\includegraphics[width=\textwidth]{figs/fig2}

\includegraphics[width=\textwidth]{figs/fig3}


\column{0.53\textwidth}

Tree vertices are grouped according to the regions that have been reached

\vspace*{1mm}

{\centerline{\textcolor{blue}{
$
\Var{vertices}(T, r) = \{ v : v \in T \wedge \Var{region}(v) = r\}
$}}}

\vspace*{2mm}

{\centerline{\textcolor{blue}{
$
\Gamma = \{r : r \in R \wedge |\Var{vertices}(T, r)| > 0\}
$}}}

\vspace*{4mm}

\textcolor{blue}{$r_\Var{from}$} is selected from
\textcolor{blue}{$\Gamma$} according to the probability distribution

\vspace*{-5mm}

\textcolor{blue}{
$$
prob(r) = \frac{1}{1 + cost^2(r)}  / \sum_{r' \in R}  \frac{1}{1 + cost^2(r')}
$$}

\vspace*{-2mm}

selection scheme balances greedy with methodical

\vspace*{9mm}

\end{columns}


\vspace*{-9mm}
{\hfill{\includegraphics[width=1.5in]{figs/figInterplaySelect}}}



\end{frame}



\begin{frame}
\frametitle{Forward Discrete Search to Compute Discrete Plans}

\begin{columns}[c]


\column{0.50\textwidth}
\includegraphics[width=\textwidth]{figs/fig4}


\column{0.48\textwidth}

Discrete plan $\sigma$ consists of a sequence of regions connecting
$r_\Var{from}$ to goal

\vspace*{3mm}

\textcolor{blue}{Exploitation} (with probability $p$)
\begin{itemize}
\item $\sigma$ computed as the shortest-path in $G=(R,E)$ from $r_\Var{from}$ to goal
\end{itemize}

\vspace*{3mm}

\textcolor{blue}{Exploration} (with probability $1 - p$)
\begin{itemize}
\item $\sigma$ computed as a random path in $G=(R,E)$ from $r_\Var{from}$ to goal
\end{itemize}

\end{columns}

\vspace*{19mm}
{\hfill{\includegraphics[width=1.5in]{figs/figInterplayPlan}}}

\end{frame}

\begin{frame}
\frametitle{\hspace*{-4mm} Sampling-based Motion Planning to Expand the Search Tree \hspace*{-4mm}}

\begin{columns}[c]


\column{0.50\textwidth}


\includegraphics[width=\textwidth]{figs/fig5}


\column{0.48\textwidth}

Expand tree of motions using the discrete plan as a guide

\vspace*{2mm}

\begin{itemize}
\item Select vertices from regions associated with the discrete plan

\item Sample input controls

\includegraphics[width=2in]{figs/figDynSim}

\item Generate new trajectories toward next region in the discrete
  plan

\item \textcolor{blue}{Allow for deviations from the
    discrete plan}
\end{itemize}

\end{columns}


\vspace*{2mm}
{\hfill{\includegraphics[width=1.5in]{figs/figInterplayExpand}}}

\end{frame}


\begin{frame}
\frametitle{{\large{Interplay: Update Discrete Plan to Reflect Motion-Planning
Progress}}}


\vspace*{-2mm}
\centerline{\includegraphics[width=0.5\textwidth]{figs/figInterplay}}

\vspace*{2mm}

\includegraphics[width=0.333\textwidth]{figs/fig3}
\includegraphics[width=0.333\textwidth]{figs/fig4}
\includegraphics[width=0.333\textwidth]{figs/fig5}

\includegraphics[width=0.333\textwidth]{figs/fig6}
\includegraphics[width=0.333\textwidth]{figs/fig7}
\includegraphics[width=0.333\textwidth]{figs/fig8}


\end{frame}

\begin{frame}
\frametitle{Experimental Results}


\includegraphics[width=1.35in]{figs/figResult1}
\movie{\includegraphics[width=0.75\textwidth]{figs/figScene1}}{movieResult1.wmv}

\vspace*{-2mm}
\href{run:movieResult1.wmv}{\footnotesize{\hfill{[movie]}}}

Compare computational efficiency to successful sampling-based motion planners
\begin{itemize}
\item RRT [LaValle, Kuffner]
\item Syclop [Plaku, Kavraki, Vardi]
\end{itemize}
running time obtained as average of 30 different
    runs

\end{frame}


\begin{frame}
\frametitle{Experimental Results (cont.)}


\includegraphics[width=1.35in]{figs/figResult2}
\movie{\includegraphics[width=0.75\textwidth]{figs/figScene2}}{movieResult2.wmv}

\vspace*{-2mm}
\href{run:movieResult2.wmv}{\footnotesize{\hfill{[movie]}}}

Compare computational efficiency to successful sampling-based motion planners
\begin{itemize}
\item RRT [LaValle, Kuffner]
\item Syclop [Plaku, Kavraki, Vardi]
\end{itemize}
running time obtained as average of 30 different
    runs

\end{frame}


\begin{frame}
\frametitle{Summary}


Approach
\begin{itemize}
\item Treats motion planning with dynamics as a search problem in a \emph{hybrid} state space composed of continuous and discrete components

\vspace*{4mm}

\item  Couples sampling-based motion planning with forward and backward discrete search over a workspace decomposition

\vspace*{4mm}

\item Obtains significant computational speedups 
\end{itemize}

\end{frame}

\end{document}




